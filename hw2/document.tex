%%%%%%%%%%%%%%%%%%%%%%%%%%%%%%%%%%%%%%%%%
% Structured General Purpose Assignment
% LaTeX Template
%
% This template has been downloaded from:
% http://www.latextemplates.com
%
% Original author:
% Ted Pavlic (http://www.tedpavlic.com)
%
% Note:
% The \lipsum[#] commands throughout this template generate dummy text
% to fill the template out. These commands should all be removed when 
% writing assignment content.
%
%%%%%%%%%%%%%%%%%%%%%%%%%%%%%%%%%%%%%%%%%

%----------------------------------------------------------------------------------------
% PACKAGES AND OTHER DOCUMENT CONFIGURATIONS
%----------------------------------------------------------------------------------------

\documentclass{article}

\usepackage{fancyhdr} % Required for custom headers
\usepackage{lastpage} % Required to determine the last page for the footer
\usepackage{extramarks} % Required for headers and footers
\usepackage{graphicx} % Required to insert images
\usepackage{lipsum} % Used for inserting dummy 'Lorem ipsum' text into the template
\usepackage{amsmath, amsfonts, bm, physics}
\usepackage{xcolor}
\usepackage{listings}
\usepackage{hyperref}
\usepackage[toc,page]{appendix}

\lstset{
    %numbers=left,
    stepnumber=1,    
    firstnumber=1,
    numberfirstline=true,
    basicstyle=\ttfamily,
    keywordstyle=\color{blue}\ttfamily,
    stringstyle=\color{red}\ttfamily,
    commentstyle=\color{green}\ttfamily,
    breaklines=true,
}

% Margins
\topmargin=-0.45in
\evensidemargin=0in
\oddsidemargin=0in
\textwidth=6.5in
\textheight=9.0in
\headsep=0.25in 

\linespread{1.1} % Line spacing

% Set up the header and footer
\pagestyle{fancy}
\lhead{\hmwkAuthorName} % Top left header
\chead{\hmwkClass\ (\hmwkClassInstructor\ \hmwkClassTime): \hmwkTitle} % Top center header
\rhead{\firstxmark} % Top right header
\lfoot{\lastxmark} % Bottom left footer
\cfoot{} % Bottom center footer
\rfoot{Page\ \thepage\ of\ \pageref{LastPage}} % Bottom right footer
\renewcommand\headrulewidth{0.4pt} % Size of the header rule
\renewcommand\footrulewidth{0.4pt} % Size of the footer rule

\setlength\parindent{0pt} % Removes all indentation from paragraphs

%----------------------------------------------------------------------------------------
% DOCUMENT STRUCTURE COMMANDS
% Skip this unless you know what you're doing
%----------------------------------------------------------------------------------------

% Header and footer for when a page split occurs within a problem environment
\newcommand{\enterProblemHeader}[1]{
  \nobreak\extramarks{#1}{#1 continued on next page\ldots}\nobreak
  \nobreak\extramarks{#1 (continued)}{#1 continued on next page\ldots}\nobreak
}

% Header and footer for when a page split occurs between problem environments
\newcommand{\exitProblemHeader}[1]{
  \nobreak\extramarks{#1 (continued)}{#1 continued on next page\ldots}\nobreak
  \nobreak\extramarks{#1}{}\nobreak
}

\setcounter{secnumdepth}{0} % Removes default section numbers
\newcounter{homeworkProblemCounter} % Creates a counter to keep track of the number of problems

\newcommand{\homeworkProblemName}{}
\newenvironment{homeworkProblem}[1][]{ % Makes a new environment called homeworkProblem which takes 1 argument (custom name) but the default is "Problem #"
  \stepcounter{homeworkProblemCounter} % Increase counter for number of
% problems
  \setcounter{homeworkSectionctr}{0}
  %\renewcommand{\homeworkProblemName}{#1} % Assign \homeworkProblemName the
% name of the problem
  \section{\arabic{homeworkProblemCounter}. #1} %
  % Make a section in the document with the
% custom problem count
  \enterProblemHeader{\homeworkProblemName} % Header and footer within the
% environment
}{
  \exitProblemHeader{\homeworkProblemName} % Header and footer after the
% environment
}

\newcommand{\problemAnswer}[1]{ % Defines the problem answer command with the content as the only argument
  \noindent\textbf{\emph{Answer: }}#1 % Just put a keyword Answer in
  % bold/italic at the beginning
}

%\newcommand{\homeworkSectionName}{}
%\newenvironment{homeworkSection}[1]{ % New environment for sections within
% homework problems, takes 1 argument - the name of the section
%  \renewcommand{\homeworkSectionName}{#1} % Assign \homeworkSectionName to the
% name of the section from the environment argument
%  \subsection{\homeworkSectionName} % Make a subsection with the custom name
% of the subsection
%  \enterProblemHeader{\homeworkProblemName\ [\homeworkSectionName]} % Header
% and footer within the environment
%}{
%  \enterProblemHeader{\homeworkProblemName} % Header and footer after the
% environment
%}

\newcounter{homeworkSectionctr}
\newenvironment{homeworkSection}{
  \medskip\noindent%         create a vertical offset to previous material
  \refstepcounter{homeworkSectionctr}% increment the environment's counter
  \textbf{(\alph{homeworkSectionctr})\ }% or \textbf,.
}

\newtheorem{theorem}{Theorem}[homeworkProblemCounter]
\newtheorem{lemma}[theorem]{Lemma}
\newtheorem{proposition}[theorem]{Proposition}
\newtheorem{corollary}[theorem]{Corollary}

\newenvironment{proof}[1][Proof]{
  \begin{trivlist}
    \item[\hskip \labelsep {\bfseries #1}]}{
  \end{trivlist}
}
\newenvironment{definition}[1][Definition]{
  \begin{trivlist}
    \item[\hskip \labelsep {\bfseries #1}]}{
  \end{trivlist}
}

\newenvironment{example}[1][Example]{
  \begin{trivlist}
    \item[\hskip \labelsep {\bfseries #1}]}{
  \end{trivlist}
}
    
\newenvironment{remark}[1][Remark]{
  \begin{trivlist}
    \item[\hskip \labelsep {\bfseries #1}]}{
  \end{trivlist}
}

\newcommand{\qed}{
  \nobreak \ifvmode \relax \else
  \ifdim\lastskip<1.5em \hskip-\lastskip
  \hskip1.5em plus0em minus0.5em \fi \nobreak
  \vrule height0.75em width0.5em depth0.25em\fi
}

\lstset{
  frame=single,
  breaklines=true,
  postbreak=\raisebox{0ex}[0ex][0ex]{\ensuremath{\color{red}\hookrightarrow\space}}
}
   
%----------------------------------------------------------------------------------------
% NAME AND CLASS SECTION
%----------------------------------------------------------------------------------------

\newcommand{\hmwkTitle}{Assignment\ \#2} % Assignment title
\newcommand{\hmwkDueDate}{Monday, April\ 11,\ 2016} % Due date
\newcommand{\hmwkClass}{STA\ 208} % Course/class
\newcommand{\hmwkClassTime}{MW 12:00 - 2:00 P.M.} % Class/lecture time
\newcommand{\hmwkClassInstructor}{Prof. James Sharpnack} % Teacher/lecturer
\newcommand{\hmwkAuthorName}{Wenhao Wu} % Your name

%----------------------------------------------------------------------------------------
% TITLE PAGE
%----------------------------------------------------------------------------------------

\title{
  \vspace{2in}
  \textmd{\textbf{\hmwkClass:\ \hmwkTitle}}\\
  \normalsize\vspace{0.1in}\small{Due\ on\ \hmwkDueDate}\\
  \vspace{0.1in}\large{\textit{\hmwkClassInstructor\ \hmwkClassTime}}
  \vspace{3in}
}

\author{\textbf{\hmwkAuthorName}}
\date{} % Insert date here if you want it to appear below your name

%----------------------------------------------------------------------------------------

\begin{document}

  \maketitle
  
  %----------------------------------------------------------------------------------------
  % TABLE OF CONTENTS
  %----------------------------------------------------------------------------------------
  
  %\setcounter{tocdepth}{1} % Uncomment this line if you don't want subsections listed in the ToC
  
  \newpage
  \tableofcontents
  \newpage
  
  %----------------------------------------------------------------------------------------
  % PROBLEM 1
  %----------------------------------------------------------------------------------------
  \begin{homeworkProblem}
    The following losses are used as surrogate losses for large margin
    classification. Demonstrate if they are convex or not, and follow the
    instructions.
    
    \begin{homeworkSection}
      exponential loss: $\phi(x)=e^{-x}$
      \vspace{10pt}
      
      \problemAnswer{
      }
    \end{homeworkSection}
    
    \begin{homeworkSection}
      truncated quadratic loss: $\phi(x)=(\max\{1-x, 0\})^2$
      \vspace{10pt}
      
      \problemAnswer{
      }
      
    \end{homeworkSection}
    
    \begin{homeworkSection}
      hinge loss: $\phi(x) = \max\{1 - x, 0\}$
      \vspace{10pt}
      
      \problemAnswer{
      }
    \end{homeworkSection}
    
    \begin{homeworkSection}
      sigmoid loss: $\phi(x) = 1 - \tanh(\kappa x)$, for fixed $\kappa > 0$
      \vspace{10pt}
      
      \problemAnswer{
      }
    \end{homeworkSection}
    
    \begin{homeworkSection}
      Plot these as a function of $x$.
      \vspace{10pt}
      
      \problemAnswer{
      }
    \end{homeworkSection}
    
    (This problem is due to notes of Larry Wasserman.)
  \end{homeworkProblem}
  %\clearpage
  
  %----------------------------------------------------------------------------------------
  % PROBLEM 2
  %----------------------------------------------------------------------------------------
  \begin{homeworkProblem}
    Consider the least-squares problem with $n$ $p$-dimensional covariates,
    $\{\mathbf{x}_i, y_i\}_{i=1}^n \subset \mathbb{R}^p \times \mathbb{R}$. We
    would like to fit the following linear model, $\hat{y}(\mathbf{x}) =
    \mathbf{x}^T\bm{\beta}$. Also, suppose that there are coefficients $C_+
    \subset \{1, \ldots , p\}$ such that for all $j \in C_+$ we require that
    $\beta_j \geq 0$, $j\in C_+$, and another set $C_- \subset \{1, \ldots,
    p\}$, such that $\beta_j \leq 0$, $j \in C_-$ (assume that $C_+$ and $C_-$
    are non-overlapping). Suppose that $\mathbf{X}^T\mathbf{X}$ is invertible.
      
    Such examples occur in insurance applications: the cost of a given insurance
    policy is based on a model for the amount of money a customer will cost the
    company, and each covariate is a variable specific to the customer (such as
    gender, age, credit history, etc.). It looks bad for the company if the
    insurance policy is more expensive for a customer that has an older account
    with the company than a newer account, when everything else is held fixed.
    Let $x_{i,j} = 1\{\mbox{customer $i$ is has had a policy for more than 2
    years}\}$, then $\beta_j \geq 0$ is necessary for this property to hold.

    \begin{homeworkSection}
      Write the constrained optimization for the empirical risk minimization
      with the constraints.
      \vspace{10pt}
      
      \problemAnswer{
      }
    \end{homeworkSection}
    
    \begin{homeworkSection}
      Derive the dual for the optimization as a function of dual parameters.
      \vspace{10pt}
      
      \problemAnswer{
      }
    \end{homeworkSection}
      
    \begin{homeworkSection}
      Write the KKT conditions and remark on the implication of the
      complementary slackness condition.
      \vspace{10pt}
      
      \problemAnswer{
      }
    \end{homeworkSection}
    
  \end{homeworkProblem}
  
  %----------------------------------------------------------------------------------------
  % PROBLEM 3
  %----------------------------------------------------------------------------------------
  \begin{homeworkProblem}
    Look at the dataset which can be found here: \url{https://archive.ics.uci.edu/ml/datasets/Blood+Transfusion+Service+Center}
      
    \begin{homeworkSection}
      You will predict the final column in the dataset, which is an indicator if
      the person has made a blood donation. Form three different kernel
      functions and $n\times n$ kernel matrices of $K_{i,j} = k(\mathbf{x}_i,
      \mathbf{x}_j)$ that you think might be appropriate. You may use 2
      different built in kernels, but must define one yourself.
      \vspace{10pt}
      
      \problemAnswer{
      }
    \end{homeworkSection}
    
    \begin{homeworkSection}
      Apply kernel SVMs and $k$-nearest neighbors (where the distance is $d(i,
      j) = k(\mathbf{x}_i, \mathbf{x}_i) + k(\mathbf{x}_j, \mathbf{x}_j) -
      2k(\mathbf{x}_i, \mathbf{x}_j))$ with each of the 3 kernels.
      \vspace{10pt}
      
      \problemAnswer{
      }
    \end{homeworkSection}
      
    \begin{homeworkSection}
      Tune any parameters basedon what you have learned about validation, and
      compare these methods with test errors (there are 6 different methods to
      compare, kNN and SVM with each kernel).
      \vspace{10pt}
      
      \problemAnswer{
      }
    \end{homeworkSection}
    
  \end{homeworkProblem}
  
  %\newpage
  %\begin{appendices} 
  %\end{appendices}
  %\bibliographystyle{unsrt}
  %\bibliography{refs}
  
  %----------------------------------------------------------------------------------------

\end{document}