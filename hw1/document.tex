%%%%%%%%%%%%%%%%%%%%%%%%%%%%%%%%%%%%%%%%%
% Structured General Purpose Assignment
% LaTeX Template
%
% This template has been downloaded from:
% http://www.latextemplates.com
%
% Original author:
% Ted Pavlic (http://www.tedpavlic.com)
%
% Note:
% The \lipsum[#] commands throughout this template generate dummy text
% to fill the template out. These commands should all be removed when 
% writing assignment content.
%
%%%%%%%%%%%%%%%%%%%%%%%%%%%%%%%%%%%%%%%%%

%----------------------------------------------------------------------------------------
% PACKAGES AND OTHER DOCUMENT CONFIGURATIONS
%----------------------------------------------------------------------------------------

\documentclass{article}

\usepackage{fancyhdr} % Required for custom headers
\usepackage{lastpage} % Required to determine the last page for the footer
\usepackage{extramarks} % Required for headers and footers
\usepackage{graphicx} % Required to insert images
\usepackage{lipsum} % Used for inserting dummy 'Lorem ipsum' text into the template
\usepackage{amsmath, amsfonts, bm}
\usepackage{xcolor}
\usepackage{listings}
\usepackage[toc,page]{appendix}

% Margins
\topmargin=-0.45in
\evensidemargin=0in
\oddsidemargin=0in
\textwidth=6.5in
\textheight=9.0in
\headsep=0.25in 

\linespread{1.1} % Line spacing

% Set up the header and footer
\pagestyle{fancy}
\lhead{\hmwkAuthorName} % Top left header
\chead{\hmwkClass\ (\hmwkClassInstructor\ \hmwkClassTime): \hmwkTitle} % Top center header
\rhead{\firstxmark} % Top right header
\lfoot{\lastxmark} % Bottom left footer
\cfoot{} % Bottom center footer
\rfoot{Page\ \thepage\ of\ \pageref{LastPage}} % Bottom right footer
\renewcommand\headrulewidth{0.4pt} % Size of the header rule
\renewcommand\footrulewidth{0.4pt} % Size of the footer rule

\setlength\parindent{0pt} % Removes all indentation from paragraphs

%----------------------------------------------------------------------------------------
% DOCUMENT STRUCTURE COMMANDS
% Skip this unless you know what you're doing
%----------------------------------------------------------------------------------------

% Header and footer for when a page split occurs within a problem environment
\newcommand{\enterProblemHeader}[1]{
  \nobreak\extramarks{#1}{#1 continued on next page\ldots}\nobreak
  \nobreak\extramarks{#1 (continued)}{#1 continued on next page\ldots}\nobreak
}

% Header and footer for when a page split occurs between problem environments
\newcommand{\exitProblemHeader}[1]{
  \nobreak\extramarks{#1 (continued)}{#1 continued on next page\ldots}\nobreak
  \nobreak\extramarks{#1}{}\nobreak
}

\setcounter{secnumdepth}{0} % Removes default section numbers
\newcounter{homeworkProblemCounter} % Creates a counter to keep track of the number of problems

\newcommand{\homeworkProblemName}{}
\newenvironment{homeworkProblem}[1][]{ % Makes a new environment called homeworkProblem which takes 1 argument (custom name) but the default is "Problem #"
  \stepcounter{homeworkProblemCounter} % Increase counter for number of
% problems
  \setcounter{homeworkSectionctr}{0}
  %\renewcommand{\homeworkProblemName}{#1} % Assign \homeworkProblemName the
% name of the problem
  \section{\arabic{homeworkProblemCounter}. #1} %
  % Make a section in the document with the
% custom problem count
  \enterProblemHeader{\homeworkProblemName} % Header and footer within the
% environment
}{
  \exitProblemHeader{\homeworkProblemName} % Header and footer after the
% environment
}

\newcommand{\problemAnswer}[1]{ % Defines the problem answer command with the content as the only argument
  \noindent\textbf{\emph{Answer: }}#1 % Just put a keyword Answer in
  % bold/italic at the beginning
}

%\newcommand{\homeworkSectionName}{}
%\newenvironment{homeworkSection}[1]{ % New environment for sections within
% homework problems, takes 1 argument - the name of the section
%  \renewcommand{\homeworkSectionName}{#1} % Assign \homeworkSectionName to the
% name of the section from the environment argument
%  \subsection{\homeworkSectionName} % Make a subsection with the custom name
% of the subsection
%  \enterProblemHeader{\homeworkProblemName\ [\homeworkSectionName]} % Header
% and footer within the environment
%}{
%  \enterProblemHeader{\homeworkProblemName} % Header and footer after the
% environment
%}

\newcounter{homeworkSectionctr}
\newenvironment{homeworkSection}{
  \medskip\noindent%         create a vertical offset to previous material
  \refstepcounter{homeworkSectionctr}% increment the environment's counter
  \textbf{(\alph{homeworkSectionctr})\ }% or \textbf,.
}

\newtheorem{theorem}{Theorem}[homeworkProblemCounter]
\newtheorem{lemma}[theorem]{Lemma}
\newtheorem{proposition}[theorem]{Proposition}
\newtheorem{corollary}[theorem]{Corollary}

\newenvironment{proof}[1][Proof]{
  \begin{trivlist}
    \item[\hskip \labelsep {\bfseries #1}]}{
  \end{trivlist}
}
\newenvironment{definition}[1][Definition]{
  \begin{trivlist}
    \item[\hskip \labelsep {\bfseries #1}]}{
  \end{trivlist}
}

\newenvironment{example}[1][Example]{
  \begin{trivlist}
    \item[\hskip \labelsep {\bfseries #1}]}{
  \end{trivlist}
}
    
\newenvironment{remark}[1][Remark]{
  \begin{trivlist}
    \item[\hskip \labelsep {\bfseries #1}]}{
  \end{trivlist}
}

\newcommand{\qed}{
  \nobreak \ifvmode \relax \else
  \ifdim\lastskip<1.5em \hskip-\lastskip
  \hskip1.5em plus0em minus0.5em \fi \nobreak
  \vrule height0.75em width0.5em depth0.25em\fi
}

\lstset{
  frame=single,
  breaklines=true,
  postbreak=\raisebox{0ex}[0ex][0ex]{\ensuremath{\color{red}\hookrightarrow\space}}
}
   
%----------------------------------------------------------------------------------------
% NAME AND CLASS SECTION
%----------------------------------------------------------------------------------------

\newcommand{\hmwkTitle}{Assignment\ \#1} % Assignment title
\newcommand{\hmwkDueDate}{Monday, April\ 4,\ 2016} % Due date
\newcommand{\hmwkClass}{STA\ 208} % Course/class
\newcommand{\hmwkClassTime}{MW 12:00 - 2:00 P.M.} % Class/lecture time
\newcommand{\hmwkClassInstructor}{Prof. James Sharpnack} % Teacher/lecturer
\newcommand{\hmwkAuthorName}{Wenhao Wu} % Your name

%----------------------------------------------------------------------------------------
% TITLE PAGE
%----------------------------------------------------------------------------------------

\title{
  \vspace{2in}
  \textmd{\textbf{\hmwkClass:\ \hmwkTitle}}\\
  \normalsize\vspace{0.1in}\small{Due\ on\ \hmwkDueDate}\\
  \vspace{0.1in}\large{\textit{\hmwkClassInstructor\ \hmwkClassTime}}
  \vspace{3in}
}

\author{\textbf{\hmwkAuthorName}}
\date{} % Insert date here if you want it to appear below your name

%----------------------------------------------------------------------------------------

\begin{document}

  \maketitle
  
  %----------------------------------------------------------------------------------------
  % TABLE OF CONTENTS
  %----------------------------------------------------------------------------------------
  
  %\setcounter{tocdepth}{1} % Uncomment this line if you don't want subsections listed in the ToC
  
  \newpage
  \tableofcontents
  \newpage
  
  %----------------------------------------------------------------------------------------
  % PROBLEM 1
  %----------------------------------------------------------------------------------------
  \begin{homeworkProblem}[(Learning paradyms)]
    Decribe the issues involved in the following learning problems using the
    terminology that we learned in the first lecture. Provide a sentence or two
    for each problem.
      
    \begin{homeworkSection}
      A `smart farm' has distributed sensors that detect moisture levels, and
      the farmers know what are the ideal moisture levels for each plant. They
      have many controls that adjust the irrigation system and they would like
      to know what settings produce the most ideal moisture levels.
      \vspace{10pt}
      
      \problemAnswer{
      
      }
    \end{homeworkSection}
    
    \begin{homeworkSection}
      Astronomers are trying to map the structure of the universe in terms of
      how galaxies cluster and form topological structures that they call
      filaments.
      \vspace{10pt}
      
      \problemAnswer{
        
      }
    \end{homeworkSection}
      
    \begin{homeworkSection}
      An online ad company wants to determine which of many ads to show each
      user based on their browser cookies.
      \vspace{10pt}
      
      \problemAnswer{
        
      }
    \end{homeworkSection}
    
    \begin{homeworkSection}
      NASA is mapping the strength of the gravitational field on the surface of
      Mars. They want you to help with determining its values in a grid of
      locations on the surface from remote sensing measurements.
      \vspace{10pt}
      
      \problemAnswer{
        
      }
    \end{homeworkSection}
  \end{homeworkProblem}
  %\clearpage
  
  %----------------------------------------------------------------------------------------
  % PROBLEM 2
  %----------------------------------------------------------------------------------------
  \begin{homeworkProblem}[(Bayes rule)]
    Consider the classification setting with features
    $\mathbf{x}\in\mathbb{R}^p$ and response $y \in \{0, 1\}$. Suppose that we
    know the joint distribution of $P(x, y)$, and the conditional
    distributions $P(y|x)$, $P(x|y)$ (an unlikely setting, but bear with me).
      
    \begin{homeworkSection}
      Under the Hamming loss, what is the true risk of a classifier $\hat{y}:
      \mathbb{R}^p \rightarrow \{0, 1\}$? Write it in terms of conditional
      distributions.
      \vspace{10pt}
      
      \problemAnswer{
      
      }
    \end{homeworkSection}
    
    \begin{homeworkSection}
      What is the Bayes rule, i.e. the classifier that minimizes the true risk?
      \vspace{10pt}
      
      \problemAnswer{
        
      }
    \end{homeworkSection}
      
    \begin{homeworkSection}
      Write in one sentence, what is the Bayes rule, as if you needed to
      describe what the Bayes risk was to someone in an elevator before you
      reached the lobby.
      \vspace{10pt}
      
      \problemAnswer{
        
      }
    \end{homeworkSection}
    
    \begin{homeworkSection}
      Prove that the Bayes risk is $1 − P(y = y^*(\mathbf{x})|\mathbf{x})$ where
      $y^*$ is the Bayes rule.
      \vspace{10pt}
      
      \problemAnswer{
        
      }
    \end{homeworkSection}
  \end{homeworkProblem}
  
  %----------------------------------------------------------------------------------------
  % PROBLEM 3
  %----------------------------------------------------------------------------------------
  \begin{homeworkProblem}[(Linear Regression)]
    Suppose that we are in the regression setting, $\mathbf{x}_i \in
    \mathbb{R}^p , y_i \in \mathbb{R}$ are $n$ pairs drawn iid, let $\mathbf{y}
    = (y_1,\ldots,y_n)$ and $\mathbf{X}^T = (\mathbf{x}_1 ,\ldots, \mathbf{x}_n
    )$, and consider the linear regression estimator
    \begin{align}
      \hat{\bm{\beta}}:= \arg\min_{\bm{\beta}\in\mathbb{R}^p} \|\mathbf{y} -
      \mathbf{X}\bm{\beta}\|_2^2.
      \label{eq:linear_regression}
    \end{align}
      
    \begin{homeworkSection}
      When is the solution to this program unique? In this case, what is the
      unique minimizer $\hat{\bm{\beta}}$?
      \vspace{10pt}
      
      \problemAnswer{
      
      }
    \end{homeworkSection}
    
    \begin{homeworkSection}
      Give an equation that the minimizers satisfy regardless of uniqueness?
      \vspace{10pt}
      
      \problemAnswer{
        
      }
    \end{homeworkSection}
      
    \begin{homeworkSection}
      Given a solution to ~\eqref{eq:linear_regression}, give a reasonable
      prediction rule $\hat{y}: \mathbb{R}^p \rightarrow \mathbb{R}$.
      \vspace{10pt}
      
      \problemAnswer{
        
      }
    \end{homeworkSection}
    
    \begin{homeworkSection}
      Suppose that $\mathbb{E}[y_i | \mathbf{x}_i ] = \mathbf{x}_i^T\bm{\beta}$
      for $i = 1,\ldots, n + 1$. For a new random draw $(\mathbf{x}_{n+1},
      y_{n+1})$, then what is the bias of $\hat{y}(\mathbf{x}_{n+1})$, i.e.
      $\mathbb{E}[\hat{y}(\mathbf{x}_{n+1}) - y_{n+1}]$?
      \vspace{10pt}
      
      \problemAnswer{
        
      }
    \end{homeworkSection}
  \end{homeworkProblem}
  
  %----------------------------------------------------------------------------------------
  % PROBLEM 4
  %----------------------------------------------------------------------------------------
  \begin{homeworkProblem}[(Simulation and ridge regression.)]
    \begin{homeworkSection}
      Simulate $\{\mathbf{x}_i\} ^n_{i=1} \subset \mathbb{R}^p$ with $p = 12$
      and $n = 200$, iid normal with mean $\mathbf{0}$ and variance
      $\mathbf{\Sigma}$ such that
      \begin{align}
        \Sigma_{j,k}=\rho^{|j-k|},\;j,k = 1,\ldots, p.
      \end{align}
      for $0 < \rho < 1$. Draw $\bm{\beta} \in \mathbb{R}^p$ such that
      $\beta_j$ are iid normal with mean 0 and variance 1, and $y_i$
      independently normal with mean $\mathbf{x}_i^T\bm{\beta}$ and variance
      1. Print out your code (not the output), which should consist of functions
      for generating these objects.
      \vspace{10pt}
      
      \problemAnswer{
        
      }
    \end{homeworkSection}
    
    \begin{homeworkSection}
      Derive an analytical expression for the solution to ridge regression,
      \begin{align}
        \hat{\bm{\beta}}:= \arg\min_{\bm{\beta}\in\mathbb{R}^p} \|\mathbf{y} -
        \mathbf{X}\bm{\beta}\|_2^2 + \lambda\|\bm{\beta}\|_2^2,
      \label{eq:linear_regression}
      \end{align}
      as a function of $\mathbf{X}, \mathbf{y}, λ$.
      \vspace{10pt}
      
      \problemAnswer{
        
      }
    \end{homeworkSection}
      
    \begin{homeworkSection}
      Provide code for solving ridge regression. Use any linear solver you like.
      \vspace{10pt}
      
      \problemAnswer{
        
      }
    \end{homeworkSection}
    
    \begin{homeworkSection}
      Set $\rho = 0.5$. Simulate the bias of $\hat{\bm{\beta}}$,
      $\|\mathbb{E}\hat{\bm{\beta}} - \bm{\beta}\|_2^2$, the variance,
      $\mathbb{E}\|\hat{\bm{\beta}} - \mathbb{E}\hat{\bm{\beta}}\|_2^2$, and the
      mean square error, $\mathbb{E}\|\hat{\bm{\beta}} - \bm{\beta}\|_2^2$, for
      many values of $\lambda$. Be sure that you see instances of overfitting
      and underfitting and can clearly see the point where $\lambda$ is optimal.
      Plot these curves as functions of $\lambda$ and include your code.
      \vspace{10pt}
      
      \problemAnswer{
        
      }
    \end{homeworkSection}
  \end{homeworkProblem}
  
  %----------------------------------------------------------------------------------------
  % PROBLEM 5
  %----------------------------------------------------------------------------------------
  \begin{homeworkProblem}[(Airfoil)]
    Download the airfoil dataset, which is linked in the homework section of the
    course site. We will focus on predicting the scaled sound pressure, which is
    the 6th row.
    \begin{homeworkSection}
      Set aside a test set at random.
      \vspace{10pt}
      
      \problemAnswer{
      
      }
    \end{homeworkSection}
    
    \begin{homeworkSection}
      Form the coefficients for ordinary least squares with the training set.
      Write a function with a new $\mathbf{x}$ and $\hat{\bm{\beta}}$ as
      arguments and returns the prediction. Use any linear solver/Cholesky
      decomposition you like.
      \vspace{10pt}
      
      \problemAnswer{
        
      }
    \end{homeworkSection}
      
    \begin{homeworkSection}
      Write a function that takes a new $\mathbf{x}, k$, and the training data,
      and outputs the k-nearest neighbor prediction with Euclidean distance.
      \vspace{10pt}
      
      \problemAnswer{
        
      }
    \end{homeworkSection}
    
    \begin{homeworkSection}
      Write a function that takes a new $\mathbf{x}$, a bandwidth parameter, and
      the training data, and outputs the kernel prediction with boxcar kernel
      and Euclidean distance.
      \vspace{10pt}
      
      \problemAnswer{
        
      }
    \end{homeworkSection}
    
    \begin{homeworkSection}
      Evaluate your methods on the test set, calculating the test error
      (empirical risk on the test set). Vary the tuning parameters and plot the
      test error as a function of the tuning parameters.
      \vspace{10pt}
      
      \problemAnswer{
        
      }
    \end{homeworkSection}
    
    \begin{homeworkSection}
      Is the best test error a good estimate of the true risk for these methods?
      Why/why not? What can be done to estimate the true risk?
      \vspace{10pt}
      
      \problemAnswer{
        
      }
    \end{homeworkSection}
  \end{homeworkProblem}
  
  %\newpage
  %\begin{appendices} 
  %\end{appendices}

  
  %----------------------------------------------------------------------------------------

\end{document}